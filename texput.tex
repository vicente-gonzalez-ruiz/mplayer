\title{MPlayer}

\maketitle
\tableofcontents

\chapter{Installation}
\begin{verbatim}
$ git clone git://source.ffmpeg.org/ffmpeg.git ffmpeg
$ cd ffmpeg
$ ./configure
$ make
\end{verbatim}

\chapter{\href{http://www.mplayerhq.hu/DOCS/HTML/en/tv-input.html}{Recording from a V4L device}}

\section{Usage examples}

\begin{verbatim}
mplayer -tv driver=v4l:width=640:height=480:outfmt=i420 -vc rawi420 -vo xv tv://
\end{verbatim}

A more sophisticated example. This makes MEncoder capture the full PAL image, crop the margins, and deinterlace the picture using a linear blend algorithm. Audio is compressed with a constant bitrate of 64kbps, using LAME codec. This setup is suitable for capturing movies.

\begin{verbatim}
mencoder -tv driver=v4l:width=768:height=576 -oac mp3lame -lameopts cbr:br=64\
    -ovc lavc -lavcopts vcodec=mpeg4:vbitrate=900 \
    -vf crop=720:544:24:16,pp=lb -o output.avi tv://
\end{verbatim}

This will additionally rescale the image to 384x288 and compresses the video with the bitrate of 350kbps in high quality mode. The vqmax option looses the quantizer and allows the video compressor to actually reach so low bitrate even at the expense of the quality. This can be used for capturing long TV series, where the video quality isn't so important.

\begin{verbatim}
mencoder -tv driver=v4l:width=768:height=576 \
    -ovc lavc -lavcopts vcodec=mpeg4:vbitrate=350:vhq:vqmax=31:keyint=300 \
    -oac mp3lame -lameopts cbr:br=48 -sws 1 -o output.avi\
    -vf crop=720:540:24:18,pp=lb,scale=384:288 tv://
\end{verbatim}


\chapter{\href{}{Recording from the X11}}

\section{Usage examples}
\begin{enumerate}

\item

\end{enumerate}

\chapter{Playing}

\chapter{Coding and transcoding}

